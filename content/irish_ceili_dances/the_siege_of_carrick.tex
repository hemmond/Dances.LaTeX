\wrapdancersnumindextrue
\irishDanceName{The Siege of Carrick}{Double Jig}{Progresivní}{Longway of sets}{4+2n}
% Name, tune, type, formation, number of dancers

{\large Progresivní double jig pro 4+2n tanečníků ve Walls of Limmerick longway formaci}\hfill

Dva páry stojí proti sobě, ženy napravo od partnerů. 
Tanec je progresivní, takže každý pár postupuje stále stejným směrem, dokud nedorazí na konec longwaye. 
Tanec se tancuje na melodii "Haste to the Wedding", ale když nevadí že se netleská do hudby, tak jde jakýkoliv double jig.

Tento tanec je uveden ve sbírce \Source{Tricitka}.

\HRule
\HRule
\vfill
\begin{description}
    \danceFigure{A}{Kolečko doleva a zpět \bars{8}}{
        \item[1-4]Kolečko sidestepem doleva, záskoky
        \item[5-8]Kolečko sidestepem doprava, záskoky
    }

    \danceFigure{B}{Hvězda za pravou a zpět \bars{8}}{
        \item[1–4]Oba páry hvězda za pravou ruku, promenádním krokem
        \item[5–8]Oba páry hvězda za levou ruku, promenádním krokem, domů
    }

    \danceFigure{C}{Back to back, jedničky \bars{8}}{
        \item[1–2]Všichni advance, první pár uvnitř, mezi druhým párem
        \item[3–4]Všichni retire, první pár venkem, kolem druhého páru
        \item[5–6]2 tlesknutí do hudby
        \item[7–8]Zátočka s partnerem za pravou ruku (hook)
    }  

    \danceFigure{D}{Back to back, dvojky + swing \bars{8}}{  
        \item[1–2]Všichni advance, druhý pár uvnitř, mezi prvním párem
        \item[3–4]Všichni retire, druhý pár venkem, kolem prvního páru
        \item[5–6]2 tlesknutí do hudby
        \item[7–8]Swing s partnerem za pravou ruku (hook) o půlku na místo druhého páru
    }
\end{description}
