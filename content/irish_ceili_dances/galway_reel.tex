\irishDanceName{Galway Reel}{Reel}{Fix}{Set}{3}
% Name, tune, type, formation, number of dancers

{\large Fixní reel pro 3 tanečníky}\hfill

Tanec ve trojici, kde pán uprostřed má po každé straně jednu dámu. 

Tento tanec je uveden ve sbírce \Source{Ztracenky}, ale autor zde čerpal z \Source{Burchenal1929}.

\HRule
\HRule
\vfill
\begin{description}
    \danceFigure{A}{Advance and return \bars{8}}{
        \item[1–4]Set promenádním krokem vpřed, čelem vzad
        \item[5–8]Set promenádním krokem vpřed, čelem vzad (na původní místa)
    }

    \danceFigure{B}{Sidestep doprava a zpět \bars{8}}{
        \item[1–4]Celý set sidestep doprava, záskoky
        \item[5–8]Celý set sidestep doleva, záskoky
    }

    \danceFigure{C}{Brány \bars{8}}{
        \item[1–2]Dámy si podají pravou ruku a na dvě promenády si prohodí místa
        \item[3–4]Dámy zvednou podané ruce a vytvoří bránu, pán na dvě promenády projde
        \item[5–8]Dtto výměna a brány (na původní místa)
    }

    \danceFigure{D}{Osmička \bars{8}}{
        \item[1–8]Pán a levá dáma se dívají na pravou dámu, pán s pravou dámou zavede osmičku (míjí se pravým ramenem) a tancují než všichni dorazí zpět na původní místa
        
    }

    \danceFigure{E}{Mixér \bars{16}}{
        \item[1–4]Celý set kolečko doprava, při záskocích pravá dáma projde bránou mezi pánem a levou dámou
        \item[5-8]Celý set kolečko doleva, při záskocích bránou prochází ta druhá dáma
        \item[9-12]Celý set kolečko doprava, při záskocích projde pán branou mezi dámami
        \item[9-12]Celý set kolečko doleva, při záskocích se set narovná do výchozí pozice (dámy budou mít prohozená místa)
    }
\end{description}  
