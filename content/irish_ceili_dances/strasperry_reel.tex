\irishDanceName{Strasperry Reel}{Reel}{Progresivní úprava}{Longway of sets}{4+2n}
% Name, tune, type, formation, number of dancers


{\large Reel s možností progrese pro 4+2n tanečníků ve Walls of Limmerick longway formaci}\hfill 
% E.g. {\large Longway for six; 3x AABB\hfill Playford, 1651}

Vždy dva páry stojí naproti sobě, pánové drží svoji partnerku ve své pravé ruce. 
Tanec je originálně fixní, takže dva páry tančí pouze spolu, ale je možné udělat úpravu tance pro progresivní

Tento tanec je uveden v knize \Source{Ztracenky}, kde autor uvádí, že zde čerpal z \Source{Orpen1931}. 

\HRule
\HRule
\vfill
\begin{description}
    \danceFigure{A}{Exchange with partner \bars{8}}{
        \item[1-2]Výměna s partnerem (míjet levým ramenem)
        \item[3-4]Dotočit se na partnera (za levým ramenem)
        \item[5-6]Výmena s partnerem (míjet pravým ramenem)
        \item[3-4]Dotočit se na sousední pár
    }

    \danceFigure{B}{Cross over and back \bars{8}}{
        \item[1-2]Promenádním krokem pass through (pánové venku - míjet pravým ramenem)
        \item[3-4]Dotočit se na sousední pár (točit se čelem k partnerovi)
        \item[5-6]Promenádním krokem pass through (pánové venku - míjet pravým ramenem)
        \item[3-4]Dotočit se na sousední pár (točit se čelem k partnerovi)
    }

    \danceFigure{C}{Hvězda bez rukou (za pravou a zpět) \bars{8}}{
        \item[1-4]Všichni promenádním krokem kolečko doleva (dáma následuje svého partnera)
        \item[5-8]Všichni promenádním krokem kolečko doprava, pán končí před cizí dámou (která je na svém místě)
    }

    \danceFigure{D}{Dance with opposite (bez záskoků)+dotočit domů \bars{8}}{
        \item[1-2]Sidestep s protějškem do pánova prava (držení za pravou ruku)
        \item[3-4]Sidestep s protějškem šikmo do pánova leva, pánové se míjí zády), končí na pánově původním místě
        \item[5-8]Swing ve kříženém drženína místě, na konci skončit na svých původních místech
    }

    \danceFigure{E}{Link arms \bars{8}}{
        \item[1-2]Pánové promenáda k sobě
        \item[3-4]Pánové zátočka za pravou ruku (loket)
        \item[5-6]Pán promenáda ke své partnerce
        \item[7-8]Zátočka za levou ruku (loket) s partnerem, skončit na místech
    }

    \danceFigure{F}{Heyes \bars{8}}{
        \item[1-2]Dámy -- výměna míst promenádním krokem, míjet pravým ramenem
        \item[3-4]Pánové -- dtto
        \item[5-6]Dámy -- dtto
        \item[7-8]Pánové -- dtto
    }
        
    \danceFigure*{PROGRESE:}{Heyes \bars{8}}{
        \item[*]Tanec původně není progresivní, ale progrese lze provést v rámci figury F
        \item[5-8]Zátočka za pravou ruku o půlku (skončit pohledem na nový pár)
    }

    \danceFigureSummary{}
\end{description}
