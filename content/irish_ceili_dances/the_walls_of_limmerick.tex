\irishDanceName{The Walls of Limerick}{Reel}{Progresivní}{Longway of sets}{4+2n}
% Name, tune, type, formation, number of dancers

{\large Progressivní reel pro 4 + 2n tanečníků -- longway vždy dvou párů naproti sobě}\hfill  

Dva páry stojí proti sobě, ženy napravo od partnerů.
Tanec je progresivní, takže každý pár postupuje stále stejným směrem, dokud nedorazí na konec longwaye.

Tento tanec je uveden ve sbírce \Source{Tricitka}, ale variantu lze nalézt i~v~\Source{OKeeffe1914}.

\HRule
\HRule
\vfill
\begin{description}
    \danceFigure{A}{Advance \& Retire \bars{8}}{
        \item[1–4]Advance \& Retire
        \item[5–8]Advance \& Retire
    }

    \danceFigure{B}{Výměny dámy, pak pánové \bars{8}}{
        \item[1–4]Dámy výměna míst sidestepem vlevo (míjet čelem), pak záskoky
        \item[5–8]Pánové výměna míst sidestepem vpravo (míjet čelem), pak záskoky
    }

    \danceFigure{C}{Dance with Opposite \bars{8}}{
        \item[1–4]Pán s protější dámou za pravou ruku sidestep \uv{ven} (do pánova leva) a~záskoky
        \item[5–8]Sidestep zpět a záskoky
    }

    \danceFigure{D}{Dance Around \bars{8}}{
        \item[1–8]Swing v kříženém držení o celé kolo, na konci koukat na další pár (zády k~předchozímu)
    }
\end{description}
