\irishDanceName{Bonfire Dance}{Reel}{Promiskuitní}{Kruhový}{6+2n}
% Name, tune, type, formation, number of dancers

{\large Promiskuitní kruhový reel, pro 6+2n tanečníků}\hfill

Tanečníci stojí v kruhu, ženy po pravici partnerů.

Tento tanec je uveden ve sbírce \Source{Tricitka}.

\HRule
\HRule
\vfill
\begin{description}
    \danceFigure{A}{Advance \& Retire (není dvojité) \bars{8}}{
        \item[1–4]Advance + Advance
        \item[5–8]Retire + Retire
    }

    \danceFigure{B}{Kolečko doprava \bars{8}}{
        \item[1–4]Sidestep doprava, záskoky
        \item[5–8]Sidestep zpět, záskoky
    }

    \danceFigure{C}{Advance \& Retire (není dvojité) \bars{8}}{
        \item[1–8]Dtto figura A
    }

    \danceFigure{D}{Kolečko doleva \bars{8}}{
        \item[1–4]Sidestep doleva, záskoky
        \item[5-8]Sidestep zpět, záskoky při kterých se otočit čelem na partnera
    }

    \danceFigure{E}{Side-step In and Out (kapky) \bars{4}}{
        \item[1–2]Každý sidestep do svého prava
        \item[3–4]Každý sidestep zpět
    }

    \danceFigure{F}{Link Arms (za pravou) \bars{4}}{
        \item[1–2]Zátočka s partnerem za pravý loket
        \item[3–4]Zátočka s partnerem za levý loket
    }

    \danceFigure{G}{Side-step In and Out (kapky) \bars{4}}{
        \item[1–4]Dtto figura E, ale každý do svého leva
    }

    \danceFigure{H}{Link Arms (za levou) \bars{4}}{
        \item[1–2]Zátočka s partnerem za levý loket
        \item[3–4]Zátočka s partnerem za pravý loket, na konci koukat zpět do kruhu
    }

    \danceFigure{I}{The Rose (dámy) \bars{16}}{
        \item[1–4]Dámy Advance + Advance, na konci se chytit do kroužku
        \item[5–8]Dámy sidestep doprava, při záskocích čelem vzad (a znovu se chytit do kroužku)
        \item[9–12]Dámy sidestep znovu do svého prava (takže zpět), záskoky
        \item[13–16]Dámy 2x Advance zpět k partnerům, kteří je podtočí pod rukou na jejich místo
    }
    
    \danceFigure{J}{The Rose (pánové) \bars{16}}{
        \item[1–16]Pánové dtto figura I, akorát sidestep doleva. Na konci opět podtáčet parnterku
    }

    \danceFigure{K}{Swing and Exchange Partners \bars{8}}{
        \item[1–4]Swing v kříženém držení o půlku (výměna míst)
        \item[5]Poklona stávajícímu partnerovi
        \item[6]Otočit se k novému partnerovi
        \item[7]Poklona novému partnerovi
        \item[8]Otočit se zpět do kruhu a chytit se
    }

    \danceFigureSummary{}
\end{description}
