\irishDanceName{Fairy Reel}{Reel}{Progresivní}{Longway of sets}{6+3n}
% Name, tune, type, formation, number of dancers

{\large Progresivní reel pro 6+3n tanečníků}\hfill

Dvě trojice (pán uprostřed) naproti sobě.
Tanec je progresivní, takže každá trojice postupuje stále stejným směrem, dokud nedorazí na konec longwaye.

Tento tanec je uveden ve sbírce \Source{Tricitka}, ale jeho verze lze najít i v \Source{Orpen1931} a \Source{OKeeffe1914}.

\HRule
\HRule
\vfill
\begin{description}
    \danceFigure{A}{Advance \& Retire \bars{8}}{
        \item[1–4]Advance \& Retire
        \item[5–8]Advance \& Retire, při druhém retire se kraje chytí
    }

    \danceFigure{B}{Kolečko doprava a zpět \bars{8}}{
        \item[1–4]Všichni kolečko sidestepem doprava, záskoky
        \item[5–8]Všichni kolečko sidestepem zpět doleva, záskoky
    }

    \danceFigure{C}{Advance \& Retire\bars{8}}{
        \item[1–8]Dtto figura A
    }

    \danceFigure{D}{Kolečko doleva a zpět \bars{8}}{
        \item[1–4]Všichni kolečko sidestepem doleva, záskoky
        \item[5–8]Všichni kolečko sidestepem zpět doprava, záskoky
    }

    \danceFigure{E}{Ledňáčci (Slip Sides) \bars{16}}{
        \item[1–4]Pán vezme dámu napravo za pravou ruku, sidestep na místo protější trojice, záskoky; volné dámy souběžně sidestep na místo druhé dámy, záskoky
        \item[5–8]Všichni sidestep zpět, záskoky (při kterých se pán otočí na druhou dámu)
        \item[9–12]Pán s dámou nalevo dtto 1-2
        \item[13–16]Dtto 3-4, pán se otočí na druhého pána
    }

    \danceFigure{F}{Pánové zátočka+Link Arms \bars{16}}{
        \item[1–4]Pánové zátočka za pravou ruku
        \item[5–8]Pánové zátočka za levou ruku
        \item[9–10]Pán zátočka s pravou dámou (za loket)
        \item[11–12]Pán zátočka s levou dámou (za loket)
        \item[13–14]Pán zátočka s pravou dámou (za loket)
        \item[15–16]Pán se vrátí domů
    }

    \danceFigure{G}{Diamant \bars{16}}{
        \item[1–4]Dámy sidestep na místo druhé dámy; pánové se natočí o 45° doleva a sidestep vpravo; všichni zakončí záskoky při kterých se otočí o 90°
        \item[5–8]Dámy sidestep stále ve stejném směru po čtverci na místo sousední dámy; pánové dtto 1-4; všichni znovu záskoky s otočkou
        \item[9-12]Dtto 1-4
        \item[13-16]Dtto 5-8
    }

    \danceFigure{H}{Brány \bars{8}}{  
        \item[1–2]Pán s dámou vpravo udělají bránu a levá jí projde (ideálně před pánem), \\obě dámy se pohybují promenádním krokem
        \item[3–4]Pán s dámou vlevo udělají bránu, pravá projde, \\jakmile pravá dáma dojde na místo, zůstává stát
        \item[5–6]Pán s pravou dámou udělají bránu, levá dáma projde
        \item[7–8]Levá dáma dojde promenádním krokem na své místo
    }

    \danceFigure{I}{Advance, Retire and Pass Through \bars{8}}{
        \item[1–4]Advance a retire
        \item[5–8]Advance a průchod protější trojicí (míjet protější osobu pravým ramenem)
    }

    \danceFigureSummary{}
\end{description}  
