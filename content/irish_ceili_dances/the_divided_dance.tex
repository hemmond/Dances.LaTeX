\irishDanceName{The Divided Dance}{Single Jig}{Progresivní}{Longway}{4+2n}
% Name, tune, type, formation, number of dancers

{\large Progresivní reel pro 4+2n tanečníků}\hfill

Partneři v páru stojí naproti sobě, všichni pánové na jedné straně (levou rukou k hudbě) a dámy naproti nim, páry se rozpočítají na první a druhý. 
Aktivní pár je ten první, druhý dělá pouze "křoví", takže je výrazně doporučuji menší počet párů v setu. 
Tanec je progresivní, takže každý pár postupuje stále stejným směrem, dokud nedorazí na konec longwaye.

Tento tanec je uveden ve sbírce \Source{Ztracenky}, ale autor zde čerpal z \Source{OKeeffe1914}.

\HRule
\HRule
\vfill
\begin{description}
    \danceFigure{A}{Cast Off and Return \bars{8}}{
        \item[1–4]1. pánové promenáda středem longwaye, 1. dámy promenáda za řadou dam
        \item[5–8]1. páry promenáda po stejné trase zpět na původní místa
    }

    \danceFigure{B}{Advance Down Centre \bars{8}}{
        \item[1–4]1. páry promenáda středem longwaye dolů, drží se za vnitřní ruce
        \item[5–8]1. páry promenáda středem longwaye nahoru na původní místa
    }

    \danceFigure{C}{Cross and circle \bars{8}}{
        \item[1–2]1. páry zátočka o půl za pravou ruku (prohodí si místa)
        \item[3–4]1. pánové pokračují 2 promenády za druhými dámami, 1. dámy dtto za prvními pány
        \item[5–8]1. pánové doputují na svá místa, 1. dámy ještě obejdou druhé dámy a pak dojdou na místo
        \item[*]Dáma musí pánovi \uv{utéct}, aby to stihla
    }

    \danceFigure{D}{Swing out \bars{8}}{
        \item[1–8]Swing o půlku (progrese)
    }
\end{description}  
