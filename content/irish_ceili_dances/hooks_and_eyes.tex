\irishDanceName{Hooks and Eyes}{Double Jig}{Progresivní}{Longway of sets}{4+2n}
% Name, tune, type, formation, number of dancers


{\large Progresivní double jig pro 4+2n tanečníků ve Walls of Limmerick longway formaci}\hfill 
% E.g. {\large Longway for six; 3x AABB\hfill Playford, 1651}

Vždy dva páry stojí naproti sobě, pánové drží svoji partnerku ve své pravé ruce.
Tanec je progresivní, takže každý pár postupuje stále stejným směrem, dokud nedorazí na konec longwaye.

Tento tanec je uveden v knize \Source{Ztracenky}, avšak autor zde čerpal z \Source{ORafferty1934} a zmiňuje také \Source{OKeeffe1914}, ale v později jmenovaném zdroji jsem tento tanec nedokázal dohledat.

\HRule
\HRule
\vfill
\begin{description}
    \danceFigure{A}{Link Arms \bars{8}}{
        \item[1-2]Partneři zátočka za pravou ruku (loket)
        \item[3-6]Partneři zátočka za levou ruku (loket)
        \item[7-8]Partneři zátočka za pravou ruku (loket) -- na svá místa
    }

    \danceFigure{B}{Kolečko promenádou doleva \bars{8}}{
        \item[1-4]Promenádním krokem kolečko doleva
        \item[5-8]Promenádním krokem kolečko doprava -- zpátky na svá místa
    }

    \danceFigure{C}{Heyes \bars{8}}{
        \item[1-2]Dámy -- výměna míst promenádním krokem, míjet pravým ramenem
        \item[3-4]Pánové -- dtto
        \item[5-6]Dámy -- dtto
        \item[7-8]Pánové -- dtto
    }

    \danceFigure{D}{Swing out \bars{8}}{
        \item[1-8]Swing o půlku -- páry výměna míst
    }
\end{description}
