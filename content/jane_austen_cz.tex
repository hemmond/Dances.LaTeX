% --- Book metadata definitions ----
\newcommand{\authorname}{JaneAusten.cz}         % Primary author (mandatory)
\newcommand{\booktitle}{Tance JaneAusten.cz}    % Main title (mandatory)
\newcommand{\subtitle}{Primárně anglické kontratance období empíru}   % Subtitle (optional)
% \newcommand{\translatorname}{}                  % Translator (optional)
\newcommand{\publisher}{JaneAusten.cz}          % Publisher name (optional)
\newcommand{\edition}{First edition}            % Edition (as text) (optional)
\newcommand{\editionyear}{2025}                 % Edition year for print run (optional)
% \newcommand{\isbn}{}                            % ISBN (optional)

% \newcommand{\publisherlogo}{
%     \includegraphics[width=0.07\linewidth]{frontmatter/logo-black.png}
% }

% --- Chapter inputs ---
\newcommand{\loadbookcontent}{
    \inputdance{content/jane_austen_cz/playford1651explanation}
    \inputdance{content/jane_austen_cz/all_in_a_garden_green}
    \inputdance{content/jane_austen_cz/blew_cap}
    \inputdance{content/jane_austen_cz/emmas_waltz}
    \inputdance{content/jane_austen_cz/the_fandango}
    \inputdance{content/jane_austen_cz/the_fandango_set3}
    \inputdance{content/jane_austen_cz/hole_in_the_wall}
    \inputdance{content/jane_austen_cz/indian_queen}
    \inputdance{content/jane_austen_cz/jamaica}
    \inputdance{content/jane_austen_cz/miss_hamiltons_reel}
    \inputdance{content/jane_austen_cz/my_lord_byrons_maggot}
    \inputdance{content/jane_austen_cz/pauls_alley}
    \inputdance{content/jane_austen_cz/rakes_of_rochester}
    \inputdance{content/jane_austen_cz/row_well_ye_mariners}
    \inputdance{content/jane_austen_cz/swedish_country_dance}
    \inputdance{content/jane_austen_cz/upon_a_summers_day}
}


\newcommand{\copyrightnotice}{  % Copyright notice (optional)
    Tance JaneAusten.cz © 2025 by Jindřich Dítě is licensed under CC BY 4.0. To view a~copy of this license, visit https://creativecommons.org/licenses/by/4.0/
}

\newcommand{\license}{    % Book license (optional)
    All content in this publication is licensed under the Creative Commons Attribution 4.0 International License. You are permitted to share, copy, and redistribute the material in any medium or format, provided that proper credit is given to the original author. For full license terms, visit \url{http://creativecommons.org/licenses/by/4.0/}.
}

\newcommand{\preface}{  % Text before TOC (optional)
    % =================
    % Preface page
    % =================
    \chapter{Úvodem}

    \lettrine{P}{rávě v ruce držíte tuto knížečku}, či si ji snad prohlížíte na monitoru počítače. Pokud čtete tento text, netuším jak se vám dostala do rukou. Možná jste na ni náhodně narazili při procházení internetu, možná jste na ni dostali odkaz když vám ji někdo potřeboval na rychlo předat, možná ji někdo vytiskl na workshop kde neměl čas (či náladu) napsat k ní předmluvu vlastní. Jak přesně se to stalo je vlastně jedno; knížečka je vám plně k dispozici. Čtěte ji jak srdce touží, sdílejte ji bez obav, k tomu je určená. Chcete-li nejnovější verzi této knížečky, najdete ji na \url{https://github.com/Adrijaned/JaneAustenDances.LaTeX}, jenom prosím, ponechte někde v knížce tento odkaz i pro ostatní :) Každopádně vám přeji mnoho úspěchů s tancem a nechť zde naleznete čeho si žádáte.

    \section*{English Country Dances - Regency Dances - Kontratance - CO TO VLASTNĚ JE?}

    Klíčovým slovem pro tance nejen období Regency je "Playford", neboli \textit{The English Dancing Master: or, Plaine and Easie Rules for the Dancing of Country Dances, with the Tune to each Dance} (Anglický taneční mistr, aneb Prostá a jednoduchá pravidla tančení venkovských tanců, s melodií ke každému tanci), kterou v letech 1651-1728 vydal John Playford, jeho synové a následovníci. Tato sbírka byla téměř do konce 19. století hlavním zdrojem tanců pro jak venkovské, tak zámecké taneční večery a plesy. Později ji následovaly sbírky vydané dalšími autory či vydavateli. V dnešní době už také existují tance, které vytvořili soudobí tanečníci ECD na staré melodie a které doplňují a  zpestřují paletu těch původních.

    Obecně se jedná o tance společenské, kde tanečníci často netancují pouze se svým partnerem, nýbrž i ostatními tanečníky okolo, to vše v prostorově zajímavých tancích sestávajících zpravidla z jednoduchých figur. Tančit lze obyčejným krokem, případně i kroky tanečními dle zdatnosti tanečníků. Velké oblibě se tyto tance těšily hlavně mezi střední třídou Anglie v 17. - 19. st., v určitých variacích jsou ale lokálně často tančené dodnes (hlavně v rámci USA).

    \section*{A trocha reklamy...}

    Pokud vás tento typ tanců zaujal a chtěli byste si někde zatančit naživo,
    \begin{description}
        \item[V Brně] funguje spolek JaneAusten.cz, který pořádá \textbf{jednou měsíčně} neformální tančírny v Brně (3h blok výuky/tance otevřený i nováčkům), nepravidelně poté empírové plesy a různé další tematické akce. Více informací je k nalezení na stránkách www.janeausten.cz a na Facebooku.
        \item[V Praze] pořádá Páv Lučištník každé druhé pondělí Kontra-Po - kontratancové pondělky podobného formátu jako neformální tančírny v Brně, jednorázově poté i další akce. Více na https://oook.cz/
    \end{description}

    \cleardoublepage   % Make sure contents page starts on right-side page
}
