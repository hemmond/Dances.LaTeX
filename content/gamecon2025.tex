% --- Specific book settings ---
\usepackage{content/jane_austen_cz/common/jane_austen_options}  % Optional, but ther should at least be .bib package.

% --- Book metadata definitions ----
\newcommand{\authorname}{JaneAusten.cz}         % Primary author (mandatory)
\newcommand{\booktitle}{Tance JaneAusten.cz}    % Main title (mandatory)
\newcommand{\subtitle}{Primárně anglické kontratance období empíru}   % Subtitle (optional)
% \newcommand{\translatorname}{}                  % Translator (optional)
\newcommand{\publisher}{JaneAusten.cz}          % Publisher name (optional)
\newcommand{\edition}{First edition}            % Edition (as text) (optional)
\newcommand{\editionyear}{2025}                 % Edition year for print run (optional)
% \newcommand{\isbn}{}                            % ISBN (optional)

% \newcommand{\publisherlogo}{
%     \includegraphics[width=0.07\linewidth]{frontmatter/logo-black.png}
% }

% --- Chapter inputs ---
\newcommand{\bookcontent}{
%     \inputdance{content/jane_austen_cz/playford1651explanation}
    \inputdance{content/jane_austen_cz/all_in_a_garden_green}
%     \inputdance{content/jane_austen_cz/blew_cap}
    \inputdance{content/jane_austen_cz/emmas_waltz}
%     \inputdance{content/jane_austen_cz/the_fandango}
%     \inputdance{content/jane_austen_cz/the_fandango_set3}
    \inputdance{content/jane_austen_cz/hole_in_the_wall}
%     \inputdance{content/jane_austen_cz/indian_queen}
    \inputdance{content/jane_austen_cz/jamaica}
%     \inputdance{content/jane_austen_cz/miss_hamiltons_reel}
    \inputdance{content/jane_austen_cz/my_lord_byrons_maggot}
    \inputdance{content/jane_austen_cz/pauls_alley}
    \inputdance{content/jane_austen_cz/rakes_of_rochester}
    \inputdance{content/jane_austen_cz/row_well_ye_mariners}
    \inputdance{content/jane_austen_cz/swedish_country_dance}
%     \inputdance{content/jane_austen_cz/upon_a_summers_day}
}

\newcommand{\loadappendix}{ %Mandatory, keep empty if not required.
}


\newcommand{\copyrightnotice}{  % Copyright notice (optional)
    Copyright \copyright{} \editionyear{} Jindra Dítě
}

\newcommand{\license}{    % Book license (optional)
    Pro GameCon 2025 sepsáno, používejte volně dle svého uvážení.
}

\newcommand{\preface}{  % Text before TOC (optional)
    % =================
    % Preface page
    % =================
    \chapter{Úvodem}

    \lettrine{T}{ento workshop s tímto názvem} se letos koná na GameConu již poněkolikáté, letos jej ale poprvé vedeme my. Doufáme, že si jej užijete/že jste si jej užili, a pro připomenutí vám dáváme k dispozici všechny tance, které jsme měli na dnešek nachystané (i když jsme je nejspíše všechny nestihli).

    \section*{English Country Dances - Regency Dances - Kontratance - CO TO VLASTNĚ JE?}

    Klíčovým slovem pro tance nejen období Regency je "Playford", neboli \textit{The English Dancing Master: or, Plaine and Easie Rules for the Dancing of Country Dances, with the Tune to each Dance} (Anglický taneční mistr, aneb Prostá a jednoduchá pravidla tančení venkovských tanců, s melodií ke každému tanci), kterou v letech 1651-1728 vydal John Playford, jeho synové a následovníci. Tato sbírka byla téměř do konce 19. století hlavním zdrojem tanců pro jak venkovské, tak zámecké taneční večery a plesy. Později ji následovaly sbírky vydané dalšími autory či vydavateli. V dnešní době už také existují tance, které vytvořili soudobí tanečníci ECD na staré melodie a které doplňují a  zpestřují paletu těch původních.

    Obecně se jedná o tance společenské, kde tanečníci často netancují pouze se svým partnerem, nýbrž i ostatními tanečníky okolo, to vše v prostorově zajímavých tancích sestávajících zpravidla z jednoduchých figur. Tančit lze obyčejným krokem, případně i kroky tanečními dle zdatnosti tanečníků. Velké oblibě se tyto tance těšily hlavně mezi střední třídou Anglie v 17. - 19. st., v určitých variacích jsou ale lokálně často tančené dodnes (hlavně v rámci USA).

    \section*{A trocha reklamy...}

    Pokud vás tento typ tanců zaujal a chtěli byste si zatančit i mimo GameCon,
    \begin{description}
        \item[V Brně] funguje spolek JaneAusten.cz, který pořádá \textbf{jednou měsíčně} neformální tančírny v Brně (3h blok výuky/tance otevřený i nováčkům), nepravidelně poté empírové plesy a různé další tematické akce (nejbližší \textbf{Letní empírová slavnost} na zámku Opočno 9. 8. 2025 a \textbf{Empírové taneční dostaveníčko} na zámku Uherčice 20. 9. 2025. Více informací je k nalezení na stránkách www.janeausten.cz a na Facebooku.
        \item[V Praze] pořádá Páv Lučištník každé druhé pondělí Kontra-Po - kontratancové pondělky podobného formátu jako neformální tančírny v Brně, jednorázově poté i další akce. Více na https://oook.cz/
    \end{description}

    Závěrem si dovolíme i pozvánku na FolCon - Již druhý ročník Brněnského festivalu hudby, tance a kultury Irska, Skotska, Bretaně, Skandinávie a balfolku. Je to něco trochu jiného, ale pokud máte čas, neváhejte se zůčastnit, určitě si to užijete!
    \clearpage
    \begin{center}
%     \includegraphics[page=1, scale=0.49]{frontmatter/Folcon plakát A4.pdf}
    Here was Folcon 2025 poster.
    \clearpage  % Folcon plakát is missing on GIT, clearpage to keep page numbering the same as already printed paper booklets. 
    \end{center}


    \cleardoublepage   % Make sure contents page starts on right-side page
}
